\section{Заключение}

В данной работе была изучена статистическая неопределённость различных сетевых структур на основе $\mathcal{E}$-меры неопределённости, предложенной в работе\cite{measures}. Предполагая, что некоторая часть наблюдений за доходностями акций распределено не нормально, а имеет $t-$распределение, или распределение Стьюдента, мы изучили поведение неопределённости структур, для который корреляция Пирсона использована в качестве меры близости. В результате серии экспериментов оказалось, что чем большая доля наблюдений имеет распределение Стьюдента, тем больше неопределённость структуры, то есть тем хуже корреляция Пирсона отражает взаимосвязь между слуайными величинами. Для тех же структур, чья мера близости основывалась на вероятности совпадения знаков, было показано экспериментально, что доля наблюдений, имеющая распределение Стьюдента, не изменяет значение статистической неопределённости. Другими словами, мера близости, основанная на вероятности совпадения знаков гораздо менее чувствительна к распределению случайных величин.

Кроме того, было проведено сравнение значений мер статистической неопределённости структур, основанных на корреляции Пирсона и на вероятности совпадения знаков. Предполагалось, что использование корреляции вероятности совпадения знаков в качестве меры близости позволит уменьшить показатели неопределённости структур, особенно при высокой доли наблюдений, имеющих распределение Стьюдента. Однако, эскперименты показали, что меры неопределённости, построенные на корреляции Пирсона практически всегда имеют меньшую неопределённость, даже если все наблюдения имеют распределение Стьюдента. В этом случае неопределённость структур, построенных на вероятности совпадения знаков, может быть немного меньше или равна.

При сравнении результатов, полученных в этой работе и результатов представленных в работе\cite{measures}, возник вопрос о влиянии начальных данных на оценку статистической неопределённости структур. Так, в работе\cite{measures} мера неопределённости для максимального остовного дерева не достигала порога $\mathcal{E}_0=0.1$ даже при 10000 наблюдениях, тогда как в ходе экспериментов в рамках данной работы, удалось достигнуть порога уже при 4000 наблюдениях.  С другой стороны, непоределённость таких сетевых структур как максимальная клика и максимальное независимое множество, оказалась гораздо выше  в данной работе, чем в работе \cite{measures} для аналогичных значений $n$, хотя значения для рыночного графа сопоставимые. Тренд особой зависимости непределённости рыночного графа от его порога $\theta$, который предполагает наибольшее значение неопределённости при средних значениях $\theta$ и меньшее при больших $\theta>0.7$ или малых $\theta<0.2$ сохраняется и в экспериментах в данной работе. При этом его можно наблюдать как при использовании корреляции Пирсона, так и при использовании вероятности совпадения знаков в качестве меры близости.

Весь исходный код для построения сетевых структур, а также для генерации наблюдений и измерения меры неопределённости доступен в репозитории на github: \url{https://github.com/ArtamonovDen/measures-of-uncertainty}.