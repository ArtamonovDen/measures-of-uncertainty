\section{Вступление}

Сегодня сетевой анализ фондовых рынков является очень активной областью исследования \cite{intro_1} \cite{intro_2} \cite{intro_3} . Фондовый рынок можно представить как полный ненаправленный взвешанный граф, в котором вершины соответствуют исследуемым акциям, а веса рёбер рассчитваются на основе некоторой меры близости, например корреляции Пирсона доходностей соответствующих акций. Таким образом, имея подобное представление в виде гарфа, мы можем использовать всю мощь теории графов для анализа и исследования закономерностей и трендов фондовых рынков, а также для составления более доходных портфелей.  Так, в работе \cite{russian_analysis} сетевой анализ был применён для анализа российского рынка ценных бумаг, а в работе \cite{chinese_analysis} - для китайского. 


Однако, полный граф представляет собой достаточно сложную структуру для анализа. Например, полный граф со 100 вершинами имеет 4950 рёбер, а граф с 500 вершинами уже 124750 рёбер. Для упрощения такой структуры, используются различные процедуры фильтрации, позволяющие преобразовать исходный граф в иную сетевую структуру с меньшим числом рёбер, которая содержит лишь наиболее значимую для анализа информацию. Такими процедурами могут быть: построение максимального остовного дерева(maximum spanning tree)\cite{mst}, рыночного графа(market graph)\cite{mg}, а также клики и независимого множества максимального размера (maximum clique and maximum independent set), построенные на основе рыночного графа. В результате применения каждой их этих процедур, мы получаем подграф исходного рыночного графа, содержащий меньшее число рёбер. При этом оставшиеся отражают наиболее сильные, относительно принятой меры близости, зависимости между акциями.


Работа с сетевыми структурами при анализе фондового рынка  должна включать оценку статистической неопределённости из-за стохастической природа временных рядов, котрыми являются данные о доходности акций. Сетевые структуры, построенные на основе данных из различных временных промежутков, могут различаться, а значит анализ подобных структур может приводить к разным выводам в зависимости от измерений. Таким образом, актуальным становится вопрос: насколько можно доверять выводам, сделанным на основе анализа той или иной структуры? 

В работе \cite{measures} была изучена статистическая неорпеделённость различных процедур фильтрации. В ней были предложены две меры для оценки статистической неопределённости сетевых структур. Относительно одной из предложенных мер, экспериментальным путём была рассчитана статистическая неопределённость различных процедур фильтрации, а также была найдена наиболее стабильная относительно статистичекой неопределённости сетевая структура. Ею оказалась максимальная клика. 

В работе \cite{measures} предполагалось, что доходности имели нормальное распределение, а мерой близости для создания сети стала выборочная корреляция Пирсона, и статистическая неопределённость была рассчитана только для таких начальных условий.

Цель моей работы - расширить результаты полученные в \cite{measures}, проведя серию экспериментов для оценки статистической неопределённости процедур фильтрации при иных начальных предположениях, а именно:
\begin{enumerate}
	\item Использование корреляции Пирсона в качестве меры близости, предполагая, что доходности акций имеют смешанное распределение нормального и распределения Стьюдента
	\item Использование вероятности совпадения знаков в качестве меры близости, предполагая, что доходности акций имеют смешанное распределение нормального и распределения Стьюдента
\end{enumerate}

Предположение о смешанном распределении позволит нам рассмотреть поведение неопределённости структур при постепенном отклонении от нормального распределения. Использование же вероятности совпадения знаков в качестве меры близости даст возможность изучить статистическую неопределённости структур  иного рода сети, основанной на иной мере близости.

Структура работы следующая. В разделе 2 мы формально определим сетевую структуру фондового рынка, детально опишем меры близости, а также процедуры фильтрации и алгоритмы для их применения. Также, будет приведён пример построения сети и применения к ней различных процедур фильтрации, основанный на реальных данных. В разделе 3 мы перейдём к рассмотрению подходов к изучению статистической неопределённости: изучим общий подход к оценке неопределённости, основанный на сравнении эталонной и выборочной сетей, определим меры неопределённости, формализуем понятия эталонной и выборочной сетей, а также понятие стабильности сетевой структуры относительно неопределённости. В разделе 4 представлены результаты экспериментов, а также их детальное описание. В последнем разделе подведём итоги и сделаем вывод и полученных результатах. 
