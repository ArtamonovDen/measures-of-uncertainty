\section{Меры неопределённости}

Построив сетевую структуру на одном наборе наблюдений, мы хотим быть уверены, что она будет актуальна и для другого набора. Для этого мы измеряем статистическую неопределённость структуры. И чем меньше эта неопределённость, тем больше мы будем  уверены, что она сохранит свой вид, а значит выводы, полученные на её основе будут более точными. 

Предполагается, что есть некотрая эталонная сетевая структура, построенная на основе параметров распределения. Например, предполагая, что случайные величины $R_1, .., R_N$ имеют совместное нормальное распределение, мы можем сказать, что их взаимоотношения описываются матрицей корреляций Пирсона $||\rho_{i j}||$. Тогда сетевая структура, полученная на основе этой матрицы будет эталонной \cite{measures}. Так как мы не знаем точно, какова матрица $||\rho_{i j}||$, мы можем оценить её матрицей выборочной корреляции $||r_{i j}||$. Тогда сетевую структуру, полученную на основе  матрицы $||r_{i j}||$ мы также будем называть эталонной и предполагать, что она отражает истинные взаимоотношения между $R_1, .., R_N$,  а любая другая сетевая структура, может приблизить эталонную структуру с той или иной точностью. 


В работе \cite{measures} был предложен единый подход к измерению неопределённости сетевых структур. Он заключается в сравнении эталонной структуры, которая строится на основе имеющихся данных, и некоторой выборочной структуры, построенной на основе данных, сгенерированных из некоторого распределения. Таким образом, мы можем изучить, насколько "сложно" сетевой структуре приблизить эталонную структуру с некоторой точностью, а именно, сколько наблюдений понадобится, чтобы получить истинную структуру с некоторым заданным порогом ошибки. Также  в \cite{measures} были предложены $\mathcal{R}$ и $\mathcal{E}$ меры близости, и на основании $\mathcal{E}$-меры были проведены эксперименты по оценке статистической неопределённости структур. В нашей работе также будет использована $\mathcal{E}$-мера.


\subsection{Мера доли ошибок($\mathcal{E}$-мера)}

$\mathcal{E}$-меру также называют мерой доли ошибок. Она основана на рассчёте количества рёбер, которые ошибочно входят или в выборочную структуру или отсутствуют в ней в сравнении с эталонной сетью.
Определяется эта мера следующим образом.

Пусть 
\begin{equation*}
  x_{1}^{i j} =
    \begin{cases}
      1, & \text{если $(i,j)$ ребро ошибочно включено в выборочную структуру,}\\
      0, & \text{в противном случае}\\
    \end{cases}       
\end{equation*}

и
\begin{equation*}
  x_{2}^{i j} =
    \begin{cases}
      1, & \text{если $(i,j)$ ребро ошибочно отсутствует в выборочной структуре,}\\
      0, & \text{в противном случае}\\
    \end{cases}       
\end{equation*}
Также определим
\[
X_1=\sum_{1\leq i \leq j \leq N}{x_{1}^{i j}}, \quad  X_2=\sum_{1\leq i \leq j \leq N}{x_{2}^{i j}}
\]

Таким образом, получается, что $X_1$ - число рёбер, которые содержит выборочная структура, но не содержит эталонная, а $X_2$ - наоборот, число рёбер, которые содержит эталонная структура, но не содержит выборочная.

Определим случайную величину $X$, такую что
\begin{equation}
X = \frac{1}{2}\left( \frac{X_1}{M_1}+\frac{X_2}{M_2} \right),
\end{equation}
где $M_1$ - максимально возможное значение $X_1$, то есть число рёбер в выборочной структуре, а $M_2$ - максимально возможное значение $X_2$, то есть число рёбер в эталонной структуре. Случайная величина $X$ принимает значение $X \in [0,1]$ и описывает общую долю ошибок. 
Тогда $\mathcal{E}(\mathcal{S}, n)$-мерой статистической неопределённости для некоторой сетевой структуры $\mathcal{S}$ будет являться математическое ожидание этой случайной величины: 
\begin{equation}
\mathcal{E}(\mathcal{S},n) = E[X]
\end{equation}

Если значение $X_1=1$, все рёбра в выборочной структуре включены неверно, если $X_2=1$, ни одно ребро из эталонной структуры не входит в выборочную. $\mathcal{E}$-мера называют $\mathcal{E}$-мерой уровня $\mathcal{E}_0$, если при числе наблюдений $n_\mathcal{E}$: $\mathcal{E}(\mathcal{S},n_\mathcal{E}) = \mathcal{E}_0$.

Говорят, что структура $\mathcal{S}_1$ стабильнее, чем структура $\mathcal{S}_2$, если $\mathcal{E}( \mathcal{S}_1, n) < \mathcal{E}(\mathcal{S}_2,n) $ для любого числа наблюдений $n$. Иными словами, статистическая непоределённость структуры $\mathcal{S}_1$  меньше, чем статистическая непоределённость структуры $\mathcal{S}_2$, если $\mathcal{E}(\mathcal{S}_1,n_1) < \mathcal{E}(\mathcal{S}_2,n_2) $, когда $n1=n2$.


