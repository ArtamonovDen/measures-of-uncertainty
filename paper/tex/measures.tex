\section{Меры неопределённости}

очередной год. очередная серия

Оценивая статистическую неопределённость сетевой струтуры, мы пытаемся ответить на вопрос: будет ли такая структура акутальной для другого набора наблюдения? Насколько изменится вид структуры, если построить её на основе меньшего числа наблюдений? Получим ли ту же самую структуру, и если нет, насколько сильно она отличается от имеющейся

Граф строится на основании некоторой информации о взаимосвязи между доходностями акций $R_1, ..., R_N$, такой как корреляция Пирсона или вероятность совпадения знаков. Делая предположение о том, что 

Оценка статистической неопределённости сетевой структуры 

В работе (ссылка) был предложен единый подход к измерению неопределённости сетевых структур. Он заключается в сравнении эталонной структуры, которая строится на основе имеющихся данных, и выборочной структуры, построенной на основе данных, сгенерированных из некоторого распределения. 


\subsection{Мера условного риска}


!!!We define the Rmeasure of statistical uncertainty of structure S (of level R0) as the number
of observations nR such that R(S, nR) = R0, where R0 is given value

тут у них Р-мера больше про число наблюдений. Поэтому надо как-то вписать УУЪУЪУЪУЪУЪ


Первый подход предполагает .. проверке гипотез

Первый подход основан на оценке условного риска совершения ошибки первого и второго рода при сравнении двух структур, а также на коэффицентах потерь, определённого для каждого из типов ошибок.   ????

ОПИСАТЬ ЗАВИСИМОСТЬ ОТ ЧИСЛА НАБЛЮДЕНИЙ!

Для заданной эталонной сетевой структуры $\mathcal{S}$ введём набор гипотез:

\begin{itemize}
	\item $h_{i j}$: ребро $(i,j)$ \textbf{не} вклчено в $\mathcal{S}$
	\item $k_{i j}$: ребро $(i,j)$ вклчено в $\mathcal{S}$
\end{itemize}


Ошибки первого и второго типа определяются следующим образом:

\begin{itemize}
	\item Ошибка I рода: ребро включено в выборочную структуру, но не включено в эталонную
	\item Ошибка II рода: ребро включено в эталонную структуру, но не включено в выборочную
\end{itemize}

Каждой из ошибок I и II рода и поставим в соответствие коэффициент потери для ребра $(i,j)$: коэффициент $a_{i j}$ для ошибки I рода и коэффициент $b_{i j}$ для ошибки II рода. Тогда для некоторой структуры $\mathcal{S}$ $\mathcal{R}$-мера неопределённости, основанная на условном риске определяется как
\begin{equation}
\mathcal{R}(\mathcal{S},n)=\sum_{1\leq i \leq j \leq N} {\left[
a_{i j}P_n(d_{k_{i j}} | h_{i j}) + b_{i j}P_n(d_{h_{i j}} | k_{i j})
\right]}
\end{equation}
где $P_n(d_{k_{i j}} | h_{i j})$ - вероятность отвергнуть гипотезу, когда она верна, а $P_n(d_{h_{i j}} | k_{i j})$ - вероятность принять гипотезу, когда она не верна.

\subsection{Мера доли ошибок}

TODO

Вторая мера неопределённости [ Eps мера? добавить в название?] основана на рассчёте количества рёбер, которые ошибочно входят или в выборочную структуру или отсутствуют в ней в сравнении с эталонной сетью.
Определяется эта мера следующим образом.

Пусть 
\begin{equation*}
  x_{1}^{i j} =
    \begin{cases}
      1, & \text{если $(i,j)$ ребро ошибочно включено в выборочную структуру,}\\
      2, & \text{в противном случае}\\
    \end{cases}       
\end{equation*}
и
\begin{equation*}
  x_{2}^{i j} =
    \begin{cases}
      1, & \text{если $(i,j)$ ребро ошибочно отсутствует в выборочной структуре,}\\
      2, & \text{в противном случае}\\
    \end{cases}       
\end{equation*}
Также определим
\[
X_1=\sum_{1\leq i \leq j \leq N}{x_{1}^{i j}}, \quad  X_2=\sum_{1\leq i \leq j \leq N}{x_{2}^{i j}}
\]
Таким образом, получается, что $X_1$ - число рёбер, которые содержит выборочная структура, но не содержит эталонная, а $X_2$ - наоборот, число рёбер, которые содержит эталонная структура, но не содержит выборочная.\\
Определим случайную величину $X$, такую что
\begin{equation}
X = \frac{1}{2}\left( \frac{X_1}{M_2}+\frac{X_2}{M_2} \right),
\end{equation}
где $M_1$ - максимально возможное значение $X_1$, то есть число рёбер в выборочной структуре, а $M_2$ - максимально возможное значение $X_2$, то есть число рёбер в эталонной структуре. Случайная величина $X$ принимает знаение $X \in [0,1]$ и описывает общую долю ошибок. 
Тогда $\mathcal{E}$-мерой статистической неопределённости для некоторой сетевой структуры $\mathcal{S}$ будет являться математическое ожидание этой случайной величины: 
\begin{equation}
\mathcal{E}(\mathcal{S},n) = E[X]
\end{equation}





TODO\\ 
Тогда математическое ожидание этой величины и будет являться  $\mathcal{E}$-мерой неопределённости: $\mathcal{E}$-мера сетевой структуры $\mathcal{S}$: $\mathcal{E}(\mathcal{S},n_\mathcal{E} = E(X))$. $\mathcal{E}$-мера называют $\mathcal{E}$-мерой уровня $\mathcal{E}_0$, если при числе наблюдений $n_\mathcal{E}$: $\mathcal{E}(\mathcal{S},n_\mathcal{E}) = \mathcal{E}_0$.\\


В статье  [ССЫЛКА] было доказано, что если $M_1$ и $M_2$ не являются случайными величинами, то  $E(\mathcal{S}, n) = \mathcal{R}(\mathcal{S}, n)$ с коэффициентом потерь при ошибках первого рода $a = \frac{1}{M_1}$ и коэффициентом потерь при ошибках второго рода $b = \frac{1}{M_2}$

TODO добавить отдельную часть про стабильность структуры. И можно наверное тут сказать что мы будем исследовать е-меру

TODO Добавить пример?

